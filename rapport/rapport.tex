%%%%%%%%%%%%%%%%%%%%%%%%%%%%%%%%%%%%%%%%%%%%%%%%
% 1. Document Class
%%%%%%%%%%%%%%%%%%%%%%%%%%%%%%%%%%%%%%%%%%%%%%%%

\documentclass[a4paper,12pt]{article}

%%%%%%%%%%%%%%%%%%%%%%%%%%%%%%%%%%%%%%%%%%%%%%%%
% 2. Packages
%%%%%%%%%%%%%%%%%%%%%%%%%%%%%%%%%%%%%%%%%%%%%%%%

\usepackage[top = 2.5cm, bottom = 2.5cm, left = 2.5cm, right = 2.5cm]{geometry} 

\usepackage[utf8]{inputenc}
\usepackage[T1]{fontenc}
\usepackage[french]{babel}
\usepackage{enumitem}
\usepackage{listings}
\usepackage{xcolor}
\usepackage{hyperref}
\usepackage{multirow}
\usepackage{booktabs}
\usepackage{graphicx} 
\usepackage{setspace}
\usepackage{mathtools}
\usepackage{amsfonts}
\usepackage{amssymb}
\usepackage{bbding}
\usepackage{xfrac}
\usepackage{datetime2}
\usepackage[absolute]{textpos}
\setlength{\parindent}{0in}
\usepackage{float}
\usepackage{fancyhdr}
\usepackage[french]{babel}



\definecolor{codegreen}{rgb}{0,0.6,0}
\definecolor{codegray}{rgb}{0.5,0.5,0.5}
\definecolor{codepurple}{rgb}{0.58,0,0.82}
\definecolor{backcolour}{rgb}{0.95,0.95,0.92}



\lstloadlanguages{Haskell}

% Define the style

\lstdefinestyle{mystyle}{
    backgroundcolor=\color{backcolour},   
    commentstyle=\color{codegreen},
    keywordstyle=\color{magenta},
    numberstyle=\tiny\color{codegray},
    stringstyle=\color{codepurple},
    basicstyle=\ttfamily\footnotesize,
    breakatwhitespace=false,         
    breaklines=true,                 
    captionpos=b,                    
    keepspaces=true,                 
    numbers=left,                    
    numbersep=5pt,                  
    showspaces=false,                
    showstringspaces=false,
    showtabs=false,                  
    tabsize=2
}


% Set the style and language
\lstset{style=mystyle, language=Haskell}
\lstset{
    inputencoding=utf8,
    extendedchars=true,
    literate={é}{{\'e}}1 {è}{{\`e}}1 {ê}{{\^e}}1 {à}{{\`a}}1 {â}{{\^a}}1 {û}{{\^u}}1 {ë}{{\"e}}1,
}

%%%%%%%%%%%%%%%%%%%%%%%%%%%%%%%%%%%%%%%%%%%%%%%%
% 3. Header (and Footer)
%%%%%%%%%%%%%%%%%%%%%%%%%%%%%%%%%%%%%%%%%%%%%%%%

\pagestyle{fancy}
\fancyhf{}
\lhead{\footnotesize IFT 3225}
\chead{\footnotesize Abdel Rahman Ibrahim, Hervé Ng'isse}
\rhead{\today}
\cfoot{\footnotesize \thepage} 

%%%%%%%%%%%%%%%%%%%%%%%%%%%%%%%%%%%%%%%%%%%%%%%%
% 4. Your document
%%%%%%%%%%%%%%%%%%%%%%%%%%%%%%%%%%%%%%%%%%%%%%%%

\begin{document}

\thispagestyle{empty}

\begin{tabular}{p{15.5cm}}
{\large \bf Technologies de l'Internet - IFT 3225}\\
Université de Montréal \\ Session : Été 2024\\ Abdel Rahman Ibrahim (20260406) \\ Hervé Ng'isse (20204609)


%%%%%%%%%%%%%%%%%%%%%%%%%%%%%%%%%%%%%%%%%%%%%%%%
% 4.1 Today's date
%%%%%%%%%%%%%%%%%%%%%%%%%%%%%%%%%%%%%%%%%%%%%%%%


\begin{textblock*}{5cm}(18.3cm,0.75cm)
   \today
\end{textblock*} \\
\hline
\end{tabular}

\vspace*{0.3cm}

\begin{center}
	{\Large \bf \underline{Rapport du travail pratique \#2}}
	\vspace{1mm}
\end{center}  

\section{Introduction}


Ce rapport détaille la répartition du travail lors de la réalisation du TP2.
Nous avons conçu une application MERN qui sert de gestionnaire de tâches 
se présentant sous la forme d'un ensemble de tuiles que l'utilisateur peut ajouter.
Chaque tuile (tâche) put être modifiée, supprimée, marquée 
comme complétée/non complétée/importante et/ou publique. 
\\ \\
Des fonctionalités de recherche, de filtrage et de pagination sont également disponibles.


\section{Routes de l'API REST}

\begin{itemize}
    \item \texttt{POST /sign-up} : Crée un nouvel utilisateur.
    \item \texttt{GET /log-in} : Connecte un utilisateur.
    \item \texttt{POST /create-task} : Ajoute une nouvelle tâche.
    \item \texttt{GET /get-all-tasks} : Récupère toutes les tâches (non publiques).
    \item \texttt{GET /get-public-tasks} : Récupère toutes les tâches publiques.
    \item \texttt{GET /get-imp-tasks} : Récupère toutes les tâches importantes.
    \item \texttt{GET /get-cmp-tasks} : Récupère toutes les tâches complètes.
    \item \texttt{GET /get-incmp-tasks} : Récupère toutes les tâches incomplètes.
    \item \texttt{GET /search-tasks} : Recherche des tâches privées.
    \item \texttt{GET /search-public-tasks} : Recherche des tâches publiques.
    \item \texttt{DELETE /delete-task/:id} : Supprime une tâche.
    \item \texttt{PUT /update-task/:id} : Met à jour une tâche.
    \item \texttt{PUT /update-imp-task/:id} : Met à jour l'importance d'une tâche.
    \item \texttt{PUT /update-cmp-task/:id} : Met à jour le statut de complétion d'une tâche.
    \item \texttt{PUT /update-public-task/:id} : Met à jour la visibilité publique d'une tâche.
\end{itemize}

\newpage

\section{Répartition du travail}

\begin{itemize}
    \item \textbf{Abdel Rahman Ibrahim} : 
    \begin{itemize}
        \item \texttt{backend}
            \begin{itemize}
                \item Implémentation de la base de données.
                \item Implémentation de la moitié des routes de l'API REST.
            \end{itemize}
        \item \texttt{frontend}
            \begin{itemize}
                \item Contribution à l'interface utilisateur.
                \item connexion, déconnexion, inscription, 
                \item ajout de tâches, recherche, filtrage, recherche.
            \end{itemize}
        \item 
        
    \end{itemize}
    \item \textbf{Hervé Ng'isse} : 
    \begin{itemize}
        \item \texttt{backend}
            \begin{itemize}
                \item Authentification et autorisation.
                \item Implémentation de l'autre moitié des routes de l'API REST.
            \end{itemize}
        \item \texttt{frontend}
            \begin{itemize}
                \item Contribution à l'interface utilisateur.
                \item Connexion, déconnexion, inscription, 
                \item Suppression de tâches, mise à jour de tâches, pagination.
            \end{itemize}
        \item 
        \end{itemize}
\end{itemize}

\section{Comment démarrer l'application}

\begin{itemize}
    \item \texttt{cd backend}
    \item \texttt{npm start}
    \item \texttt{cd frontend}
    \item \texttt{npm start}
\end{itemize}


\end{document}
